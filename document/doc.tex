\documentclass[a4j,12pt]{jarticle}
\usepackage{listings}

\title{eJS compiler 内部設計と実装}
\author{片岡 崇史}
\date{\today}


\begin{document}
\maketitle


\section{用語の定義}

\begin{description}
\item{ejsc} 本資料で説明するコンパイラ.JavaScriptのソースをeJSVMの
  バイトコード形式に変換する.
\end{description}



\section{コンパイル規則}

$C[[e]]\ \rho\ d $ は,環境$\rho$における,JavaScriptのコード
$e$のコンパイル規則である.
コンパイルしたコードの実行結果を$d$に格納する.
\footnotesize{
\begin{tabbing}
xx\=xx\=xx\=\kill \\
\textbf{数値} \\
ただし,\textit{value\_of}($n$)は,$n$のバイトコードにおける
数値表現である.\\
\>$ C[[n]]\ \rho\ d\ = $ \\
\>\>\texttt{number}\  $d$\  \textit{value\_of}($n$) \\
\\
\textbf{文字列} \\
ただし,\textit{name\_of}($str$)は,$str$のバイトコードにおける
文字列表現である.\\
\>$ C[[str]]\ \rho\ d\ = $ \\
\>\>\texttt{string}\ $d$\ \textit{name\_of}($str$) \\
\\
\textbf{真偽値,undefined,null} \\
ただし,\textit{const\_of}($val$)は,$val$のバイトコードにおける
文字列表現である.\\
\>$ C[[val]]\ \rho\ d\ =$ \\
\>\>\texttt{specconst}\ $d$\ \textit{const\_of}($val$) \\
\\

\textbf{変数} \\
\>$C[[x]]\ \rho\ d\ =$ \\
\>\>\textit{if}($\rho$($x$)\ $==$\ \textit{local}($l$, $o$))\ \{ \\
\>\>\>\texttt{getlocal}\ $d$\ $l$\ $o$ \\
\>\>\}\ \textit{else\ if}($\rho$($x$)\ $==$\ \textit{arg}($l$, $o$))\ \{ \\
\>\>\>\texttt{getarg}\ $d$\ $l$\ $o$ \\
\>\>\}\ \textit{else}\ \{ \\
\>\>\>\texttt{string}\ $r$\ \textit{name\_of}($x$) \\
\>\>\>\texttt{getglobal}\ $d$\ $r$ \\
\>\>\} \\
\\

\textbf{代入} \\
  ただし,$A[[x]]\ \rho\ s$は環境$\rho$における$x$に対して$s$の値を代入する. \\
\>$A[[x]]\ \rho\ s\ =$\\
\>\>\textit{if}($\rho$($x$)\ $==$\ \textit{local}($l$, $o$))\ \{ \\
\>\>\>\texttt{setlocal}\ $d$\ $l$\ $o$ \\
\>\>\}\ \textit{else\ if}($\rho$($x$)\ $==$\ \textit{arg}($l$, $o$))\ \{ \\
\>\>\>\texttt{setarg}\ $d$\ $l$\ $o$ \\
\>\>\}\ \textit{else}\ \{ \\
\>\>\>\texttt{string}\ $r$\ \textit{name\_of}($x$) \\
\>\>\>\texttt{setglobal}\ $d$\ $r$ \\
\>\>\} \\
\\

\textbf{変数への代入} \\
\>$C[[x\ \texttt{=}\ e]]\ \rho\ d =$ \\
\>\>$C[[e]]\ \rho\ d$ \\
\>\>$A[[x]]\ \rho\ d$ \\
\\

\textbf{\texttt{.}を用いたプロパティへのアクセス} \\
\>$C[[e\texttt{.}x]]\ \rho\ d\ f\ =$ \\
\>\>$C[[e]] \rho\ t_1\ False$ \\
\>\>\texttt{string}\ $t_2$\ \textit{name\_of}($x$) \\
\>\>\texttt{getprop}\ $d$\ $t_1$\ $t_2$ \\
\\

\textbf{\texttt{.}を用いたプロパティへの代入} \\
\>$C[[e_1\texttt{.}x\ \texttt{=}\ e_2]]\ \rho\ d\ f\ =$ \\
\>\>$C[[e_1]] \rho\ t_1\ False$ \\
\>\>\texttt{string}\ $t_2$\ \textit{name\_of}($x$) \\
\>\>$C[[e_2]] \rho\ t_3\ False$ \\
\>\>\texttt{setprop}\ $t_1$\ $t_2$\ $t_3$ \\
\\

\textbf{\texttt{[]}を用いたプロパティへのアクセス} \\
\>$C[[e_1\texttt{[}e_2\texttt{]}]]\ \rho\ d\ f\ =$ \\
\>\>$C[[e_1]] \rho\ t_1\ False$ \\
\>\>$C[[e_2]] \rho\ t_2\ False$ \\
\>\>\texttt{getprop}\ $d$\ $t_1$\ $t_2$ \\
\\

\textbf{\texttt{[]}を用いたプロパティへの代入} \\
\>$C[[e_1\texttt{[}e_2\texttt{]}\ \texttt{=}\ e_3]]\ \rho\ d\ f\ =$ \\
\>\>$C[[e_1]] \rho\ t_1\ False$ \\
\>\>$C[[e_2]] \rho\ t_2\ False$ \\
\>\>$C[[e_3]] \rho\ t_3\ False$ \\
\>\>\texttt{setprop}\ $t_1$\ $t_2$\ $t_3$ \\
\\

\textbf{if文} \\
\>$C[[\texttt{if(}e\texttt{)}\ s1\ \texttt{else}\ s2]]$ \\
\>\>$C[[e]]\ \rho\ t_1\ False$ \\
\>\>\texttt{jumpfalse}\ $t_1$\ \textgt{L1} \\
\>\>$C[[s1]]\ \rho\ d\ False$ \\
\>\>\texttt{jump}\ \textgt{L2} \\
\>\>\textgt{L1}: \\
\>\>$C[[s2]]\ \rho\ d\ False$ \\
\>\>\textgt{L2}: \\
\\

\textbf{for文} \\
\>$C[[\texttt{for(}e_1\texttt{;}e_2\texttt{;}e_3\texttt{)}\ s_1]]\ \rho\ d\ f\ =$\\
\>\>$C[[e_1]]\ \rho\ d\ False$ \\
\>\>\texttt{jump}\ \textgt{L1} \\
\>\>\textgt{L2:} \\
\>\>$C[[s_1]]\ \rho\ d\ False$ \\
\>\>$C[[e_1]]\ \rho\ d\ False$ \\
\>\>\textgt{L1:} 
\>
\\

\textbf{レシーバの無い関数呼び出し} \\
FLは現在の環境でのレジスタウィンドウの大きさを表している. \\
\>$C[[func\texttt{(}x_1, ... , x_n\texttt{)}]]\ \rho\ d\ f\ =$ \\
\>\>$C[[func]]\ \rho\ t_f\ False$ \\
\>\>$C[[x_1]]\ \rho\ t_1\ False$ \\
\>\>$C[[x_2]]\ \rho\ t_2\ False$ \\
\>\>$...$ \\
\>\>$C[[x_n]]\ \rho\ t_n\ False$ \\
\>\>\textit{if}($f$)\{ \\
\>\>\>\texttt{move}\ \texttt{\$2}\ $t_1$ \\
\>\>\>\texttt{move}\ \texttt{\$3}\ $t_1$ \\
\>\>\>$...$ \\
\>\>\>\texttt{move}\ \texttt{\$n+1}\ $t_1$ \\
\>\>\>\texttt{tailcall}\ $t_f$\ $n$ \\
\>\>\>\texttt{geta}\ $d$ \\
\>\>\}\ \textit{else}\ \{ \\
\>\>\>\texttt{move}\ \$($FL-n+1$)\ $t_1$ \\
\>\>\>\texttt{move}\ \$($FL-n+2$)\ $t_2$ \\
\>\>\>$...$ \\
\>\>\>\texttt{move}\ \$($FL$)\ $t_n$ \\
\>\>\>\texttt{call}\ $t_f$\ $n$ \\
\>\>\>\texttt{setfl}\ FL \\
\>\>\>\texttt{geta}\ $d$ \\
\>\>\} \\
\\

\textbf{レシーバのある関数呼び出し} \\
\>$C[[e.m\texttt{(}x_1, ... , x_n\texttt{)}]]\ \rho\ d\ f\ =$ \\
\>\>$C[[e]]\ \rho\ t_e\ False$ \\
\>\>\texttt{string}\ $t_ms$\ \textit{name\_of}($m$) \\
\>\>\texttt{getprop}\ $t_m$\ $t_e$\ $t_ms$ \\
\>\>$C[[x_1]]\ \rho\ t_e\ False$ \\
\>\>$C[[x_2]]\ \rho\ t_1\ False$ \\
\>\>$...$ \\
\>\>$C[[x_n]]\ \rho\ t_{n+1}\ False$ \\
\>\>\textit{if}($f$)\{ \\
\>\>\>\texttt{move}\ \texttt{\$1}\ $t_e$ \\
\>\>\>\texttt{move}\ \texttt{\$2}\ $t_1$ \\
\>\>\>$...$ \\
\>\>\>\texttt{move}\ \texttt{\$n+1}\ $t_n$ \\
\>\>\>\texttt{tailsend}\ $t_f$\ $n$ \\
\>\>\>\texttt{geta}\ $d$ \\
\>\>\}\ \textit{else}\ \{ \\
\>\>\>\texttt{move}\ \$($\textrm{FL}-n$)\ $t_e$ \\
\>\>\>\texttt{move}\ \$($\textrm{FL}-n+1$)\ $t_1$ \\
\>\>\>$...$ \\
\>\>\>\texttt{move}\ \$(FL)\ $t_n$ \\
\>\>\>\texttt{send}\ $t_m$\ $n$ \\
\>\>\>\texttt{setfl}\ FL \\
\>\>\>\texttt{geta}\ $d$ \\
\>\>\} \\
\\

\textbf{関数式} \\
\>$C[[\texttt{function(}arg\texttt{)\{}\ s\ \texttt{\}}\ 
  locals\ :=\ \{x\ \mid\ sの中で宣言された変数名x\}]]\ \rho\ d\ =$ \\
\>\>\texttt{makeclosure}\ $d$\ $F[[s]]\ arg\cup locals$\ $d$ \\
\\

\textbf{関数からのリターン} \\
ただし,\textit{const\_of}($val$)は,$val$のバイトコードにおける
文字列表現である.\\
\>$ C[[\texttt{return}\ e]]\ \rho\ d\ =$ \\
\>\>$C[[e]]\ \rho\ d $ \\
\>\>\texttt{seta}\ $d$ \\
\>\>\texttt{ret} \\

\end{tabbing}
}





\section{コンパイラの全体像}
ejscは,以下の4つの部分から構成される.
\begin{enumerate}
  \item Antlrによって生成された字句解析器と構文解析器
  \item 1.が生成したASTをESTree形式に変換する変換器
  \item ESTree形式のASTをiASTに変換する変換器
  \item iAST形式のASTからバイトコードを生成するコード生成器
\end{enumerate}

まず,入力のJavaScriptのソースコードを字句解析,構文解析し,
ParseTreeを生成する.
この解析はAntlrによって生成された解析器を使用する.

ESTree生成器は,ParseTreeからESTree形式のASTを生成する.
ESTreeは,JavaScriptの有名なASTの一つである.
しかし,ESTreeはeJSVMのバイトコード生成は困難であるので,
ESTreeから独自のASTの形式に変換する.
このASTをiASTと呼ぶことにする.
ESTreeからiASTへ変換する前には,ESTreeの一部を変形する処理を行う.
詳細は後に説明する.

最後に,コード生成器がeJSVMのバイトコードを生成する.


\section{ESTree上での変形}

ESTreeの変形について述べる.

ここで行うことは二点ある.
一つは,ESTree内にある関数宣言をスコープの先頭に移動することである.
移動した関数宣言は,後でESTreeからiASTに変換する際に,ローカル変数に
関数オブジェクトを代入する意味のASTに変換される.
これによって,関数を宣言したスコープ内でのみ,
その関数を呼び出すことができるようにする.

もう一つは,名前付き関数式の除去である.
ESTreeに現れる名前付き関数式のノードを同じ意味になるように
無名関数式から成るノードに変換する.
変換するのは関数式のみであり,関数宣言を変形しない.

% 具体的には...

\subsection{実装}

関数宣言をスコープの先頭に移動するための実装は,ESTreeNormalizerクラス
の内部クラスであるHoistingFuncDeclクラスにされている.
FunctionDeclarationとFunctionExpressionのbodyは新しいスコープになるので,
bodyの直下のノードにある全てのFunctionDeclarationをbodyの先頭に
列挙する.

名前付き関数式を除去するための実装は,ESTreeNormalizerクラスの
内部クラスであるEliminatingNamedFuncクラスにされている.
ESTreeに,FunctionExpressionのノードでそのnameプロパティがnullでない
(つまり,無名関数式)があれば,これを変換する.




\section{iASTの生成}

ESTreeからiASTを生成する.
入力となるESTreeは,先に行われるESTreeを変形する処理によって,
関数宣言がスコープの先頭にあることと,名前付き関数が無いことが
保証されている.

iASTとESTreeの間で大きく異なる点は,
変数宣言と関数宣言が無いことである.
変数宣言時に初期化代入する文は,代入式のみ残る.
変数宣言と関数宣言が無い代わりに,
FunctionExpressionにlocalsプロパティを持たせている.
localsプロパティは,FunctionExpressionのbody内で宣言される
変数と関数の名前を列挙している.

% 例を入れる

% ESTree -> iAST への変換規則を書く

% hoistDeclarations



\section{コード生成}

\subsection{コード生成に関わるデータ}

\subsubsection{BCode}

バイトコード命令の基底クラスである.
全ての命令の定義はこのクラスの派生クラスである.

\subsubsection{Register}

レジスタ番号となる値のみを持つクラスである.

\subsubsection{Fl}

フレーム長となる値のみを持つクラスである.
setfl命令の定義となるISetflクラスは,Flクラスのインスタンスを
プロパティに持つ.

\subsubsection{Label}

BCodeをプロパティに持ち,これでジャンプ先の命令を指定する.
例えば,jump命令の定義であるIJumpクラスは,ジャンプ先を指定するために,
ジャンプ先のBCodeを持つLabelをプロパティに持つ.

\subsubsection{JSLabel}

\texttt{break}文と\texttt{continue}文が記述可能な文
(For文,ForIn文,While文,DoWhile文,SwitchCase文)の中で,
\texttt{break}文や\texttt{continue}文があったときにジャンプする
先を示す.
詳しくは後で.

\subsubsection{Environment}

あるノードをコンパイルするときに与える環境を持つクラスである.






\lstinputlisting[numbers=left,label=example01]{./example01.js}

\end{document}
